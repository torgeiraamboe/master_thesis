

\section{Overview of the thesis}

\textit{\Cref{ch:1}} is spent on defining DG-algebras, seeing some examples and  defining formality for DG-algebras. The latter half of it is designated to introducing the general theory which allows us to have a homotopy theory for DG-algebras, called model categories---as well as proving that DG-algebras admit such a theory. Together these two parts should answer what we meant by ``sufficiently simple'' and ``has access to'' in the introduction.

\textit{\Cref{ch:2}} is an interlude into the ``information'' part of the central question. We introduce certain algebraic operations---called Massey products---that serves as ``information'' in the DG-algebra. We then prove that this information is not accessible to the cohomology algebra, and that all Massey products must vanish if the DG-algebra is formal. 

\textit{\Cref{ch:3}} is another interlude, this time into transferring algebraic structures between objects. This chapter is not an integral part of the thesis, but  it is meant to give strong intuition into how the algebraic objects we introduce in \ref{ch:4} behave. In this chapter we try to deform a DG-algebra by a deformation retraction, to see if the result is still a DG-algebra. 

\textit{\Cref{ch:4}} introduces a new algebraic object called $\A$-algebras. These objects generalize the DG-algebras we develop in \cref{ch:1}. This added generalization gives a better behaved homotopy theory, as well as making the central question nice and easy to state and answer. In the second part of the third chapter we prove a new---as far as the author is aware---result, introducing a special case where if the Massey products we developed in \cref{ch:2} all vanish, we must have a formal DG-algebra. 

\textit{\Cref{ch:5}} is then spent trying to find an interesting example to the new result we proved at the end of \cref{ch:3}. We show that a certain class of topological spaces---those with Lusternik-Schnirelmann category 1---satisfy the requirements of the theorem, and hence that they must be formal spaces. 

Lastly, we have two appendices. 

\textit{\Cref{ap:A}} features two long proofs that were omitted during the thesis in order to not break the flow of reading. The first proves that the deformed DG-algebra from \Cref{ch:3} is not associative, but associative up to homotopy. The second proves that a certain decomposition of a DG-algebra gives us a deformation retraction onto its cohomology algebra.

\textit{\Cref{ap:B}} is added to showcase an alternative method to constructing the model structure on the category of DG-algebras. This alternative construction uses monoids in monoidal model categories. 


