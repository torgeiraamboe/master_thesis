

\section{Overview of the thesis}

\textit{The first chapter} is spent on defining DG-algebras, seeing some examples and  defining formality for DG-algebras. The latter half of it is designated to introducing the general theory which allows us to have a homotopy theory for DG-algebras, as well as proving that DG-algebras admit such a theory. Together these two parts should answer what we meant by ``sufficiently simple'' and ``has access to'' in the introduction.

\textit{The second chapter} is an interlude into the ``information'' part of the central question. We introduce certain algebraic operations - called Massey products- that serves as ``information'' in the DG-algebra. We then prove that this information is not accessible to the cohomology algebra, and that all such ``information'' must be trivial if the DG-algebra is ``sufficiently simple''. 

\textit{The third chapter} introduces a new algebraic object called $\A$-algebras. These objects generalize the DG-algebras we develop in chapter 1. This added generalization gives a better behaved homotopy theory, as well as making the central question nice and easy to state and answer. In the second part of the third chapter we prove a new---as far as the author is aware---result, introducing a special case where the Massey products we developed in chapter 2 being trivial, actually means that the DG-algebra is ``sufficiently simple''. 

\textit{The fourth and final chapter} is then spent trying to find an interesting example to the new result we proved at the end of chapter 3. We show that a certain class of topological spaces satisfy the requirements of the theorem, and hence that they must be ``sufficiently simple spaces''. 

Lastly, we have added an appendix to showcase an alternative method to constructing the model structure on the category of DG-algebras. This alternative construction uses monoids in monoidal categories. 


