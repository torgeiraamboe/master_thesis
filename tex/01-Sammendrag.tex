

\section{Sammendrag}

En differensialgradert algebra (DG-algebra) kan ofte sees på som en samling av høyt detaljert informasjon om et topologisk rom. To slike tilfeller er algebraen ac kokjeder, og kohomologiringen. Den sistnevnte er ofte lett å regne ut, men den førstnevnte innholder generelt mye mer informasjon om rommet vi er interessert i. I denne avhandlingen utforsker vi forholdet mellom disse to algebraene---mer presist noen situasjoner hvor disse to innholder den samme homotopiske informasjonen. Slike algebraer kalles formelle DG-algebraer. 

For å forstå hvilken type homotopisk informasjon en DG-algebra kan inneholde, konstruerer vi hindringer til formalitet gjennom høyere kohomologioperasjoner. Vi generaliserer så DG-algebraer til $\A$-algebraer, og ser på noen måter å bruke denne generaliserte teorien som et felles rammeverk for både DG-algebraer, og høyere kohomologioperasjoner. Dette rammeverket tillater oss å vise at en viss klasse av topologiske rom---nemlig de med Lusternik-Schnirelmann kategori 1---har formelle kokjede algebraer. 