

\section{Sammendrag}

En \textit{differensialgradert algebra} (DG-algebra) kan ofte sees på som en samling av høyt detaljert topologisk informasjon. To slike tilfeller er algebraen av kokjeder, og kohomologiringen til et topologisk rom. Den sistnevnte er ofte lett å regne ut, men den førstnevnte innholder generelt mye mer informasjon om rommet vi er interessert i. I denne avhandlingen utforsker vi forholdet mellom disse to algebraene---mer presist noen situasjoner hvor disse to innholder den samme homotopiske informasjonen. Slike algebraer kalles \textit{formelle} DG-algebraer. 

For å forstå hvilken type homotopisk informasjon en DG-algebra kan inneholde, konstruerer vi hindringer for formalitet gjennom høyere ordens kohomologioperasjoner---kalt \textit{Massey produkter}. Vi generaliserer så DG-algebraer til \textit{$\A$-algebraer}, og ser på noen måter å bruke denne generaliserte teorien som et felles rammeverk for både DG-algebraer og Massey produkter. Dette rammeverket tillater oss å vise at en viss klasse av topologiske rom---nemlig de med Lusternik-Schnirelmann kategori 1---har formelle kokjede algebraer. 