
\section{Intrinsic formality}

\begin{definition}{Intrinsically formal DG algebra}

A DG algebra $(A, d)$ is called intrinsically formal if any other DG algebra with cohomology algebra isomorphic to $H(A)$ is formal. 
\end{definition}

This definition, as with the definition of being formal, also stems from Sullivan and rational homotopy theory. As we mentioned earlier, rational homotopy theory studies the torsion free part of the regular homotopy theory of spaces, and is a nicer, more simple version of regular homotopy theory. In this field intrinsic formality is connected to the question of how many rational homotopy types can have the same cohomology ring. This can be nice to keep in mind, that these sort of problems we are dealing with during this thesis, often have topological implications. 
\todo{Write so that it makes sense with the previous historical stuff}