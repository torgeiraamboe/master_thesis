

\section{The homotopy theory of DG-algebras}

When lookin at some examples of model categories, we mentioned that the model structure on the category of DG-algebras would be similar to the model structure on chain complexes of modules over a ring. 

The model structure is constructed by Jardine in \cite{jardine} and is described by the weak equivalences being the quasi-isomorphisms, the fibrations being the degreewise surjections and the cofibrations being all maps that have the left lifting property with respect to acyclic fibrations. 

Lets prove that this is in fact a model structure on the category $DGA_k$ of DG-algebras over a field $k$. This construction and proof holds more generally for $k$ a commutative unital ring. Note that much of this proof is directly inspired by the  original paper \cite{jardine}, but we have tried to fill in some details, and prove some parts that is left out by Jardine. We will not prove that the second factorization in \textbf{MC4} holds in $DGA_k$, as it requires us to go into the so-called ``small objects argument''. For a proof of that result, see \cite[Theorem 2.1.14]{hovey}, and for a more in depth conceptual treatment, see \cite{small_objects}.

To help us with some of the proofs of the different axioms we let:
\begin{itemize}
	\item $S(x)$ be the free graded $k$-algebra on one generator $x$ in degree $n$ together with the differential defined by $d(x)=0$.
	
	\item $T(x)$ be the free graded $k$-algebra on two generators, $x$ and $dx$, together with the differential defined by $d(x)=dx$ and $d(dx)=0$. This is the free DG-algebra on one generator.
	\item $C(x)$ be the free cochain complex on one generator $x$ in degree $n$, i.e. the complex 
\begin{equation*}
    C(x)^i = 
    \begin{cases}
        &  0, i\neq n, i\neq n+1 \\
        & k, i = n, i= n+1
    \end{cases}
\end{equation*}
where the differential is trivial, except for being the identity on $k$ in degree $n$. 
\end{itemize}


The coproduct of two DG-algebras $A$ and $B$ is defined by $A\ast B = T(A\otimes B)/I$, where $T(A\otimes B)$ is the tensor algebra
\begin{equation*}
    T(A\otimes B) = \bigoplus_{n \in \N}(A\otimes B)^{\otimes n}
\end{equation*}
and $I$ is the ideal generated by 
\begin{equation*}
    \begin{cases}
        & (a\otimes b_1)\otimes (1\otimes b_2) - a\otimes b_1 b_2    \\
        & (a_1\otimes 1)\otimes (a_2 \otimes b) - a_1 a_2 \otimes b
    \end{cases}
\end{equation*}

Note that we can identify $T(x)$ with the tensor algebra on $C(x)$, i.e. 
\begin{equation*}
    T(x) \overset{\cong}\longrightarrow T(C(x)) = \bigoplus_{i\geq 0}C(x)^{\otimes i}
\end{equation*}

Hence we have 
\begin{equation*}
H^i(T(x)) = 
\begin{cases}
    &k, i=0 \\
    &0, else
\end{cases}
\end{equation*}

\begin{definition}
Let $A$ be a DG-algebra and $C$ a cochain complex. We define the DG-algebra $A[C]$ to be the cochain complex
\begin{equation*}
    A[C] = A\oplus (A\otimes C\otimes A) \oplus (A\otimes C \otimes A \otimes C \otimes A) \oplus \cdots
\end{equation*}
together with multiplication defined by 
\begin{equation*}
    (a_1\otimes b_1 \otimes \cdots \otimes b_k \otimes a_{k+1}) \cdot (a'_1\otimes b'_1 \otimes \cdots \otimes b'_l \otimes a'_{l+1}) = a_1 \otimes \cdots \otimes b_k \otimes a_{k+1}a'_1 \otimes \cdots \otimes b'_l \otimes a'_{l+1}
\end{equation*}
\end{definition}

Any map from this DG-algebra $f:A[C]\longrightarrow B$ is uniquely determined by its restriction to its first component $A$ and the chain map on the first occurring $C$, i.e. the map $j_f$ defined by by the composition
\begin{equation*}
    j_f: C\overset{inc}\longrightarrow A\otimes C\otimes A \subseteq A[C]\overset{f}\longrightarrow B
\end{equation*}
where $inc(c) = 1\otimes c \otimes 1$. 

Hence we have an isomorphism $A\ast_k T(x) \cong A[C(X)]$ from the coproduct to this ``free algebra'' on $C$. This is because $T(x)\cong T(C(x))$, and the map $A[C(x)]\rightarrow A\ast_k T(C(x))$ is uniquely determined by sending $A$ into $A$, and $C(x)$ into $C(x)$ as a component of $T(C(x))$. 


\begin{lemma}
The map $k\longrightarrow T(x)$ is a cofibration. 
\end{lemma}
\begin{proof}
    We need to show that a lift $h:T(X)\longrightarrow A$ exists for all commuting diagrams
\begin{center}
\begin{tikzcd}[column sep=large, row sep=large]
	{k} & {A} \\
	{T(x)} & {B}
	\arrow[from=1-1, to=1-2]
	\arrow[from=1-1, to=2-1]
	\arrow["{g}", from=2-1, to=2-2]
	\arrow["{f}", from=1-2, to=2-2]
\end{tikzcd}
\end{center}
    
The push-out of the diagram 
\begin{center}
\begin{tikzcd}[column sep=large, row sep=large]
	{k} & {A} \\
	{T(x)}
	\arrow[from=1-1, to=1-2]
	\arrow[from=1-1, to=2-1]
\end{tikzcd}    
\end{center}
is the coproduct $A\ast_k T(x)$, which we know is isomorphic to $A[C(x)]$. Hence we have a unique map $B\longrightarrow A[C(x)]$ by the universal property of a push-out, i.e. 
\begin{center}
\begin{tikzcd}[column sep=large, row sep=large]
	{k} & {A} \\
	{T(x)} & {B} \\
	&& {A[C(x)]}
	\arrow[from=1-1, to=1-2]
	\arrow[from=1-1, to=2-1]
	\arrow[from=1-2, to=2-2]
	\arrow[from=2-1, to=2-2]
	\arrow[from=1-2, to=3-3, bend left]
	\arrow[from=2-1, to=3-3, bend right]
	\arrow["{\phi}", from=2-2, to=3-3, dashed]
\end{tikzcd}    
\end{center}
We can then define the map $h = p_A \circ \phi \circ g$, where $p_A$ is the projection onto the first component, which is uniquely determined by being the identity on $A$ and $j_{p_A} = 0$. 
    
Then the diagram 
\begin{center}
\begin{tikzcd}[column sep=large, row sep=large]
	{k} & {A} \\
	{T(x)} & {B} \\
	&& {A[C(x)]}
	\arrow[from=1-1, to=1-2]
	\arrow[from=1-1, to=2-1]
	\arrow["{f}", from=1-2, to=2-2]
	\arrow["{g}"', from=2-1, to=2-2]
	\arrow[from=1-2, to=3-3, bend left]
	\arrow[from=2-1, to=3-3, bend right]
	\arrow["{\phi}", from=2-2, to=3-3]
	\arrow["{h}", from=2-1, to=1-2]
	\arrow["{p_A}"', from=3-3, to=1-2, bend right=60]
\end{tikzcd}
\end{center}
commutes everywhere, and we are done. 
\end{proof}




\begin{theorem}
The category $DGA_k$ of DG-algebras over a field $k$, together with the three classes of morphisms; $W$, $C$, $F$, as described above, form a model category.
\end{theorem}

\begin{proof}
We need to check the four axioms. 

\textbf{MC 1:} This point consists of three sub-points. We first prove that $F$ is retraction closed, then $W$ and finally $C$. 

Assume $f:A\longrightarrow B$ is a retract of $g:X\longrightarrow Y$ where $g\in F$. This means we have a diagram

\begin{center}
\begin{tikzcd}[column sep=large, row sep=large]
	{A} & {X} & {A} \\
	{B} & {Y} & {B}
	\arrow["{k}"', from=1-1, to=1-2]
	\arrow["{i}"', from=1-2, to=1-3]
	\arrow["{h}", from=2-2, to=2-3]
	\arrow["{j}", from=2-1, to=2-2]
	\arrow["{f}"', from=1-1, to=2-1]
	\arrow["{g}"', from=1-2, to=2-2]
	\arrow["{f}"', from=1-3, to=2-3]
	\arrow["{id_A}", from=1-1, to=1-3, bend left]
	\arrow["{id_B}"', from=2-1, to=2-3, bend right]
\end{tikzcd}   
\end{center}

Let $b$ be a homogeneous element in degree $n$. Want to show that there is a homogeneous element $a$ such that $f(a)=b$, as this would show that $f$ is degree-wise surjective, i.e. $f\in F$. 
    
Let $y = j(b)$. Since $g\in F$ it is a degree-wise surjection and hence there exists an element $x\in X$ such that $g(x)=y$. Let $i(x)=a$. Then we have 
\begin{align*}
    f(a) 
    &= f(i(x) \\
    &= h(g(x)) \\
    &= h(j(b)) \\
    &= id_B(b) = b.
\end{align*}
which shows that the retract of a fibration is again a fibration. 

For the second part we let $g\in W$ and $f$ still a retraction of $g$. We have the same retraction diagram as above, which induces the following diagram in cohomology:
\begin{center}
\begin{tikzcd}[column sep=large, row sep=large]
	{H(A)} & {H(X)} & {H(A)} \\
	{H(B)} & {H(Y)} & {H(B)}
	\arrow["{\overline{k}}"', from=1-1, to=1-2]
	\arrow["{\overline{i}}"', from=1-2, to=1-3]
	\arrow["{\overline{h}}", from=2-2, to=2-3]
	\arrow["{\overline{j}}", from=2-1, to=2-2]
	\arrow["{\overline{f}}"', from=1-1, to=2-1]
	\arrow["{\overline{g}}"', from=1-2, to=2-2]
	\arrow["{\overline{f}}"', from=1-3, to=2-3]
	\arrow["{id_{H(A)}}", from=1-1, to=1-3, bend left]
	\arrow["{id_{H(B)}}"', from=2-1, to=2-3, bend right]
\end{tikzcd}   
\end{center}

Recall that we want to show that $f$ induced an isomorphism in cohomology. As $g$ is surjective, we can use the same argument as above---when we had $f\in F$---to get that $\overline{f}$ is surjective. Assume now that $\overline{f}([a]) = \overline{f}([a'])$. Then $\overline{j}(\overline{f}([a])) = \overline{j}(\overline{f}([a']))$, which means $\overline{g}(\overline{k}([a])) = \overline{g}(\overline{k}([a']))$. But $\overline{g}$ is an isomorphism, and hence we have $\overline{k}([a]) = \overline{k}([a'])$ and finally
\begin{equation*}
    [a] = id_A([a]) = \overline{i}(\overline{k}([a])) = \overline{i}(\overline{k}([a'])) = id_A([a']) = [a']
\end{equation*}
This shows that $\overline{f}$ is both injectiv and surjective, i.e. an isomorphism---which means that $f\in W$. 

For the last part we let $g\in C$, and $f$ still a retraction of $g$. Recall that we need to have a lift of $f$ with respect to all acyclic fibrations $[p\colon U\longrightarrow V]\in F\cap W$, i.e. the existence of the dotted morphism $\phi$ in the following diagram
\begin{center}
\begin{tikzcd}[column sep=large, row sep=large]
A \arrow[d, "f"'] \arrow[rr, "s"]           &  & U \arrow[d, "p"] \\
B \arrow[rr, "r"'] \arrow[rru, "\phi", dotted] &  & V               
\end{tikzcd}
\end{center}

We get this by producing a lift from $g$, as it is in $C$. As $f$ is a retraction of $f$ we can extend the above diagram to 
\begin{center}
\begin{tikzcd}[column sep=large, row sep=large]
A \arrow[d, "f"'] \arrow[rrr, "s", bend left] \arrow[r, "k"'] & X \arrow[d, "g"'] \arrow[r, "i"] & A \arrow[d, "f"] \arrow[r, "s"] & U \arrow[d, "p"] \\
B \arrow[rrr, "r"', bend right] \arrow[r, "j"]                & Y \arrow[r, "h"']                & B \arrow[r, "r"']               & V               
\end{tikzcd}
\end{center}

This diagram has the sub-diagram  
\begin{center}
\begin{tikzcd}[column sep=large, row sep=large]
X \arrow[d, "g"'] \arrow[rr, "s\circ i"]      &  & U \arrow[d, "p"] \\
Y \arrow[rr, "r\circ h"'] \arrow[rru, "\psi"] &  & V               
\end{tikzcd}
\end{center}
where we know that the lift $\psi$, exists, as $g\in C$ and $p\in F\cap W$. We can then define $\psi = \psi\circ j$, i.e. the dotted arrow in the following diagram
\begin{center}
\begin{tikzcd}[column sep=large, row sep=large]
A \arrow[d, "f"'] \arrow[rrr, "s", bend left] \arrow[r, "k"]               & X \arrow[d, "g"'] \arrow[r, "i"]              & A \arrow[d, "f"] \arrow[r, "s"] & U \arrow[d, "p"] \\
B \arrow[rrr, "r"', bend right] \arrow[r, "j"'] \arrow[rrru, "\phi", dotted] & Y \arrow[r, "h"'] \arrow[rru, "\psi", dotted] & B \arrow[r, "r"']               & V               
\end{tikzcd}
\end{center}
Hence all retractions of morphisms in $C$ satisfy the lifting property with respect to morphisms in $F\cap W$, which means they are again in $C$. 


\textbf{MC 2:} Isomorphisms of DG-algebras have the two out of three property, and since quasi-isomorphisms are defined by inducing isomorphisms on the cohomology algebras, they also satisfy the property. 
    
\textbf{MC 3:} Notice that one half of this axiom holds by definition, as we defined cofibrations to be the morphisms that satisfied the left lifting property with respect to acyclic fibrations. For the other half we need to show that if we have a diagram 
\begin{center}
\begin{tikzcd}[column sep=large, row sep=large]
A \arrow[d, "i"'] \arrow[rr, "f"] &  & X \arrow[d, "p"] \\
B \arrow[rr, "g"']                &  & Y               
\end{tikzcd}
\end{center}
where $i$ is an acyclic cofibration, and $p$ a fibration, then a lift exists. 
    
We can translate the problem to showing that every morphism $i\in C\cap W$ has the right lifting property w.r.t. all fibrations $p\in F$. 
    
Assume we have proven MC4, then we can factorize $i= f\circ j$ such that $j\in C\cap W$ and $f\in F$. Since two of them are in $W$ then the last one also is by MC2.

Hence we have a diagram 
\begin{center}
\begin{tikzcd}[column sep=large, row sep=large]
	{A} & {\overline{B}} \\
	{B} & {B}
	\arrow["{j}", from=1-1, to=1-2]
	\arrow["{f}", from=1-2, to=2-2]
	\arrow["{i}"', from=1-1, to=2-1]
	\arrow["{id_B}"', from=2-1, to=2-2]
	\arrow["{h}", from=2-1, to=1-2, dotted]
\end{tikzcd}
\end{center}
    
that has a lift $h$ by the definition of $C$. 

By the commutativity of the previous diagram we have a new diagram 
\begin{center}
\begin{tikzcd}[column sep=large, row sep=large]
	{A} & {A} & {A} \\
	{B} & {\overline{B}} & {B}
	\arrow["{id_A}"', from=1-1, to=1-2]
	\arrow["{id_A}"', from=1-2, to=1-3]
	\arrow["{f}", from=2-2, to=2-3]
	\arrow["{h}", from=2-1, to=2-2]
	\arrow["{i}"', from=1-1, to=2-1]
	\arrow["{j}"', from=1-2, to=2-2]
	\arrow["{i}"', from=1-3, to=2-3]
	\arrow["{id_A}", from=1-1, to=1-3, bend left]
	\arrow["{id_B}"', from=2-1, to=2-3, bend right]
\end{tikzcd}   
\end{center}
which means that $i$ is a retraction of $j$, which we know has the right lifting property w.r.t all maps $f\in F$, hence $i$ does as well.



\textbf{MC 4:} Let $f:A\longrightarrow B$ be any map between two DG-algebras. We can form the factorization 
\begin{equation*}
    A\overset{i}\longrightarrow A\ast (\ast_{b\in B} T(B))\overset{p}\longrightarrow B
\end{equation*}
where $i$ is the inclusion and $p$ is the map that sends the generator $b \in T(B)$ to $b\in B$. The map $q$ is a fibration as it is  degreewise surjection, and the map $i$ is a filtered colimit of maps $A\longrightarrow T(b_1)\ast \cdots T(b_n)\ast A$ which all are acyclic cofibrations by iterating the construction of the isomorphism $T(x)\ast A \cong A[C(x)]$. Hence it is also itself an acyclic cofibration. 

The last factorization is as mentioned left out, due to us not covering the small objects argument in this thesis. See \cite[Lemma 3]{jardine} for a proof using this argument. 
\end{proof}


As we now know that $DGA_k$ is a model category, we know there exists a terminal and an initial object. In $DGA_k$ the terminal object is $0$---the complex consisting only of zeroes with only trivial differentials and trivial multiplication---while the initial object is the ground field $k$---treated as a DG-algebra by having only one copy of $k$ in degree zero and zeroes everywhere else. 

Since the unique map $0: A\longrightarrow 0$ is a degreewise surjection for any DG-algebra $A$ we know that all DG-algebras are fibrant objects in this model structure. 




\section{More formality}

This new framework allows us to reconsider the definition of a formal DG algebra. We take the category $DGA_k$, which we now know is a model category with $W$ being the collection of quasi-isomorphisms, and produce its homotopy category $HoDGA_k = DGA_k[W^{-1}]$. We can then define a DG algebra to be formal as follows. 

\begin{definition}[Formal DG-algebra]
A DG-algebra $(A, d)$ is called formal if it is isomorphic to its cohomology algebra $H(A)$ in $HoDGA_k$.
\end{definition}

This is the precise reason we referred to formal DG-algebras in the abstract as being the algebras that contain the same homotopical information as their cohomology algebra. They are isomorphic in the homotopy category, hence contain the same information up to homotopy. Such information is what we call homotopical information. 

Unfortunately not all isomorphisms in $HoDGA_k$ come from a single quasi-isomorphism in $DGA_k$. Quasi-isomorphisms in $DGA_k$ are not invertible, and not even homotopy invertible in general. Thus a zig-zag of quasi-isomorphisms is the best we can hope for, which means that this new definition is precisely the same as the old. 

A feature of this new framework is that we actually don't need an arbitrary zig-zag of quasi-isomorphisms, we only need a single span. To prove this we will need the following. 

\begin{definition}[Right proper model category]
Let $\C$ be a model category. We say it is right proper if the pullback of a weak equivalence along a fibration is again a weak equivalence. 
\end{definition}

As a consequence of the fact that all objects in $DGA_k$ are fibrant, and the fact that pullbacks of weak equivalences along fibrations between fibrant objects is again a weak equivalence, we have that $DGA_k$ - with the model structure defined above - is a right proper model category.  
% https://ncatlab.org/nlab/show/proper+model+category

\begin{theorem}
\label{thm:span}
Let $A$ and $B$ be quasi-isomorphic DG-algebras. Then there exists a DG-algebra $C$ and two quasi-isomorphisms $q_1, q_2$ such that $A\overset{q_1}\longleftarrow C \overset{q_2}\longrightarrow B$
\end{theorem}
\begin{proof}
Since we know $A$ and $B$ are quasi-isomorphic, we have a zig-zag of quasi-isomorphisms between them. Assume this sequence has length $r$. There are four possible ways this zig-zag can look at its ends. On the left we can have either a quasi-isomorphism $A\longrightarrow A_1$, or $A\longleftarrow A_1$. Similarly on the other end we can have either $A_r\longrightarrow B$ or $A_r\longleftarrow B$. 

Notice that if we prove that we can turn a cospan $A_{i-1}\longrightarrow A_i\longleftarrow A_{i+1}$ into a span $A_{i-1}\longleftarrow C_i \longrightarrow A_{i+1}$, then we have proved the theorem. This is due to composition of quasi-isomorphisms again being quasi-isomorphisms. Hence at the ends we get for example
\begin{center}
\begin{tikzcd}[column sep=large, row sep=large]
A \arrow[r, "q_0"] & A_1                                      & A_2 \arrow[l, "q_1"'] \arrow[r, "q_2"] & A_3  & \cdots \arrow[l] \\
                   & C_1 \arrow[lu, "p_0"] \arrow[ru, "p_1"'] &                                        &               &       
\end{tikzcd}    
\end{center}
which by composing $p_2$ and $q_2$ reduces the length $r$ zig-zag to a length $r-1$ zig-zag. Hence we can do this for all the cospans in the zig-zag, and the end result will be a simple nice span. 

So, lets assume we have a cospan $A_{i-1}\overset{q_{i-1}}\longrightarrow A_i\overset{q_i}\longleftarrow A_{i+1}$. We could take its pullback, but we have no justification for saying that the maps in the pullback are again quasi-isomorphisms. Since $DGA_k$ is right proper we know that pullback of a weak equivalence along a fibration is again a weak equivalence. By \textbf{MC4} we know that the quasi-isomorphisms $q_{i-1}$ and $q_i$ can be factorized as $q_{i-1} = i \circ q$ and $q_i = j\circ p$, where $q, p$ are cofibrations and $i, j$ are acyclic fibrations. By the two-out-of-three property of weak equivalences, we also know that $q, p$ are acyclic cofibrations. This means we have the following diagram
\begin{center}
\begin{tikzcd}[column sep=large, row sep=large]
C_1 \arrow[r, "i"]                           & A_i & C_2 \arrow[l, "j"']                       \\
A_{i-1} \arrow[ru, "q_{i-1}"] \arrow[u, "q"] &     & A_{i+1} \arrow[lu, "q_i"] \arrow[u, "p"']
\end{tikzcd}
\end{center}

Consider now the diagram
\begin{center}
\begin{tikzcd}[column sep=large, row sep=large]
A_{i-1} \arrow[d, "q"'] \arrow[r, "id_{A_{i-1}}"] & A_{i-1} \arrow[d] \\
C_1 \arrow[r]                                     & 0                
\end{tikzcd}
\end{center}
As all objects are fibrant we know that $A_{i-1}\longrightarrow 0$ is a fibration. By \textbf{MC3}  there exists a lift, $q'\colon C_1\longrightarrow A$ making the sub-diagrams commute. In particular $q'\circ q = id_{A_{i-1}}$. Since $q$ and $id_{A_{i-1}}$ are both quasi-isomorphisms, $q'$ has to be as well due to the two-out-of-three property. Similarily a quasi-isomorphism $p'$ exists such that $p'\circ p = id_{A_{i+1}}$. 

The next step is to take the pullback of $C_1\overset{i}\rightarrow A_i\overset{j}\leftarrow C_2$, i.e.
\begin{center}
\begin{tikzcd}[column sep=large, row sep=large]
C_3 \arrow[d, "j'"'] \arrow[r, "i'"] & C_2 \arrow[d, "j"] \\
C_1 \arrow[r, "i"]                   & A_i               
\end{tikzcd}    
\end{center}
As $i$ and $j$ are both acyclic fibrations, the maps $i'$ and $j'$ will be as well. This is because being a fibration is stable under pullback, and being a quasi-isomorphism is stable when we are pulling back along a fibration, which both maps are. Now we have a diagram

\begin{center}
\begin{tikzcd}[column sep=large, row sep=large]
                                                           & C_3 \arrow[ld, "j'"'] \arrow[rd, "i'"] &                                                     \\
C_1 \arrow[r, "i"] \arrow[d, "q'"', bend right]            & A_i                                    & C_2 \arrow[l, "j"] \arrow[d, "p'", bend left]       \\
A_{i-1} \arrow[ru, "q_{i-1}"'] \arrow[u, "q"', bend right] &                                        & A_{i+1} \arrow[lu, "q_i"] \arrow[u, "p", bend left]
\end{tikzcd}    
\end{center}

And the compositions $q'\circ i'$ and $p'\circ j'$ are both quasi-isomorphisms, as all four individually are. Hence we have a span $A_{i-1}\longleftarrow C_3 \longrightarrow A_{i+1}$, which by the argument above means we can reduce any zig-zag of quasi-isomorphisms $A\leftarrow \cdots \rightarrow B$ to a single span $A\leftarrow C\rightarrow B$. 
\end{proof}





We don't actually use the specific model structure on $DGA_k$ in this proof, so this holds in for general right proper model categories where all objects are fibrant. 

Notice that this also implies that $DGA_k$ satisfies the right Ore condition, i.e. that given a cospan $A\overset{a}\rightarrow C \overset{q}\leftarrow B$, where $q$ is a quasi-isomorphism, then there exists a span $A\overset{q'}\leftarrow C' \rightarrow B$ where $q'$ is a quasi-isomorphism. This follows from the proof above as we can factorize $a$ into $f\circ c$, where $c$ is an acyclic cofibration and $f$ a fibration. We take the pullback of $D\rightarrow C\leftarrow B$ to get the diagram
\begin{center}
\begin{tikzcd}[column sep=large, row sep=large]
C' \arrow[dd, "q'", bend right] \arrow[r, "a'"] & B \arrow[d, "q"] \\
A \arrow[r, "a"] \arrow[d, "c"]                 & C                \\
D \arrow[ru, "f"']                              &                 
\end{tikzcd}    
\end{center}
where now $q'$ is a quasi-isomorphism as it is a pullback of one along the fibration $f$. As in the proof above, because $A$ is fibrant we get a left inverse to the acyclic cofibration $c$, which is again a quasi-isomorphism. This is the lift we get from the following diagram
\begin{center}
\begin{tikzcd}[column sep=large, row sep=large]
A \arrow[d, "c"'] \arrow[r, "id_{A}"] & A \arrow[d] \\
D \arrow[r]                                     & 0                
\end{tikzcd}
\end{center}
Denote this lift by $c'$. We now have the final diagram
\begin{center}
\begin{tikzcd}[column sep=large, row sep=large]
C' \arrow[dd, "q'", bend right] \arrow[r, "a'"] & B \arrow[d, "q"] \\
A \arrow[r, "a"] \arrow[d, "c"']                & C                \\
D \arrow[ru, "f"'] \arrow[u, "c'"', bend right] &                 
\end{tikzcd}    
\end{center}
where we by the composition $c'\circ q'$, which is a quasi-isomorphism, we get our wanted span $A\overset{c'\circ q'}\leftarrow C'\rightarrow B$. 

This Ore condition allows the homotopy category $HoDGA_k$ to be described even more explicitly using spans as the morphisms. This means that a morphism $A\longrightarrow B$ is given by a span
\begin{center}
\begin{tikzcd}[column sep=large, row sep=large]
  & C \arrow[ld, "q"'] \arrow[rd] &   \\
A &                               & B
\end{tikzcd}
\end{center}
where $q$ is a quasi-isomorphism. Composition of morphisms $A\longrightarrow B$ and $B\longrightarrow C$ is then given as
\begin{center}
\begin{tikzcd}[column sep=large, row sep=large]
  &                                   & D_3 \arrow[ld, "q_3"'] \arrow[rd] &                                   &   \\
  & D_1 \arrow[ld, "q_1"'] \arrow[rd] &                                   & D_2 \arrow[ld, "q_2"'] \arrow[rd] &   \\
A &                                   & B                                 &                                   & C
\end{tikzcd}
\end{center}
where $q_3$ exists due to the Ore condition. As $q_3\circ q_1$ is again a quasi-isomorphism, we have a new span
\begin{center}
\begin{tikzcd}[column sep=large, row sep=large]
  & D_3 \arrow[rd] \arrow[ld, "q_3\circ q_1"'] &   \\
A &                                            & C
\end{tikzcd}.
\end{center}



% \begin{definition}{Homotopy}
% Let $f, g:A\longrightarrow B$ be two DG morphisms. A homotopy $h:f\rightarrow g$ is a map $h:A\longrightarrow B$ such that 
% \begin{itemize}
%     \item $g-f = d_B h + h d_A$
%     \item $h m_A = m_B(f\otimes h + h\otimes g)$
% \end{itemize}
% \end{definition}
% This is definition 4.3. in Proute french thesis
