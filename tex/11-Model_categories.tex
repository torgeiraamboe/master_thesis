
\section{Model categories}

We now turn our eye to the homotopy theory of DG-algebras. Homotopy theories are contained in structures called model categories, which are additional structures, called model structures, on categories. Having such a model structure on a category, allows us to define the notion of homotopy, which again allows us to define homotopy equivalences, and the other homotopical constructions we are used to from the homotopy theory of topological spaces. 

\begin{definition}[Retraction]
    We say a map $f:A\longrightarrow B$ is a retract, or a retraction of a a map $g:X\longrightarrow Y$ if there exists a commuting diagram 
\begin{center}
\begin{tikzcd}[column sep=large, row sep=large]
	{A} & {X} & {A} \\
    {B} & {Y} & {B}
    \arrow["{f}", from=1-1, to=2-1]
    \arrow["{g}", from=1-2, to=2-2]
    \arrow["{f}", from=1-3, to=2-3]
    \arrow[from=1-1, to=1-2]
    \arrow[from=1-2, to=1-3]
    \arrow[from=2-1, to=2-2]
    \arrow[from=2-2, to=2-3]
    \arrow["{Id_A}", from=1-1, to=1-3, bend left]
    \arrow["{Id_B}"', from=2-1, to=2-3, bend right]
\end{tikzcd}
\end{center}
such that the horizontal maps compose to the identity. 
\end{definition}

\begin{definition}[Retraction closed]
A class of morphisms $R$ is called retraction closed if every retraction of a morphism in $R$ is again in $R$. 
\end{definition}

Note that if a retraction closed class contains all the identitiy morphism, then it must contain all isomophims as well. 



\begin{definition}[Model structure]
Let $\mathcal{C}$ be a category. A model structure on $\mathcal{C}$ is a choice of three distinguished collections of maps, $F$, $C$ and $W$, in $\mathcal{C}$
such that the axioms below hold. The maps in the collections are called fibrations, cofibrations and weak equivalences respectively, and maps in $F\cap W$ are called acyclic fibrations and maps in $C\cap W$ are called acyclic cofibrations. 

\textbf{MC1:}
Any retraction\footnote{Definition is in the preliminaries} of a map in one of the three classes is again in the same class, i.e. all three classes are retraction closed. 

\textbf{MC2:}
The collection $W$ of weak equivalences has the two out of three property, i.e. if two out of $f, g, g\circ f$ is a weak equivalence, then the third is as well.

\textbf{MC3:}
If we have a commutative square 

\begin{center}
\begin{tikzcd}[column sep=large, row sep=large]
    A \arrow[r, "f"] \arrow[d, "i"']         & X \arrow[d, "p"] \\
    B \arrow[r, "g"] \arrow[ru, "h", dotted] & Y               
\end{tikzcd}
\end{center}

where either $i\in C$ and $p\in F\cap W$, or $i\in C\cap W$ and $p\in F$, then there exists a lift $h$ making both subdiagrams commute.

\textbf{MC4:}
Given any map $f:X\longrightarrow Y$ in $\mathcal{C}$ we can factor it as $f=p\circ i$, where $p\in F$ and $i\in C\cap W$ and as $f=p'\circ i'$ where $i'\in C$ and $p'\in F\cap W$. 
\end{definition}

We then define a model category\index{Model category} to be a bicomplete category - a category where all small limits and colimits exists - with a model structure. 

The two parts in \textbf{MC3} are often stated as fibrations having the right lifting property with respect to acyclic cofibrations, and cofibrations having the left lifting property with respect to acyclic fibrations. 


The archetypal example of a model category is the category of topological spaces. This category has two often used model structures, often called the Quillen (or Serre) model structure and the Strøm model structure. 

\begin{example}[Quillen model structure on topological spaces]
The underlying category is the category of topological spaces with continuous maps. The fibrations consists of the Serre fibrations, which are maps that have the so called homotopy lifting property with respect to all CW complexes. This property is described by lifts $h$ existing when we have a CW complex $X$ and a diagram
\begin{center}
\begin{tikzcd}[column sep=large, row sep=large]
	{A} & {X} \\
	{A\times I} & {Y}
	\arrow["{i}"', from=1-1, to=2-1]
	\arrow["{f}", from=1-1, to=1-2]
	\arrow["{p}", from=1-2, to=2-2]
	\arrow["{g}"', from=2-1, to=2-2]
	\arrow["{h}", from=2-1, to=1-2, dotted]
\end{tikzcd}
\end{center}
where $I$ is the unit interval and $i_0:A\longrightarrow A\times I$ is the inclusion into the first component. 

The cofibrations are defined by being induced by retracts of relative CW complexes, i.e. maps $g:X\longrightarrow Y$ where $Y$ is made from $X$ by attaching cells. The weak equivalences are weak homotopy equivalences, which are maps that induce isomorphisms on all homotopy groups. 
\end{example}

\begin{example}[Strøm model structure on topological spaces]
As with the previous example, the category of interest is the category of topological spaces with continuous maps. The fibrations are the Hurewicz fibrations, which satisfies the homotopy lifting property with respect to all topological spaces, not just the CW complexes as the Serre fibrations. The cofibrations are the closed Hurewicz cofibrations, which satisfy the homotopy extension property. This property is described by a lift $h$ existing when the diagram below commutes.

\begin{center}
\begin{tikzcd}[column sep=large, row sep=large]
	{A} & {Y^I} \\
	{B} & {Y}
	\arrow["{i}"', from=1-1, to=2-1]
	\arrow["{f}", from=1-1, to=1-2]
	\arrow["{p}", from=1-2, to=2-2]
	\arrow["{g}"', from=2-1, to=2-2]
	\arrow["{h}", from=2-1, to=1-2, dotted]
\end{tikzcd}
\end{center}

Here $Y^I$ is the path space of $Y$ and $p$ is the projection onto the start of a path. We call a map $i:A\longrightarrow B$ satisfying this property a Hurewicz cofibration, and we say it is a closed Hurewicz cofibration if its image is closed in $B$. The weak equivalences are given by the homotopy equivalences, i.e. the maps that are invertible up to homotopy.  
\end{example}

\begin{example}
Another example is the category of chain complexes of modules over some ring, $Ch(R-mod)$. Here the model structure consists of quasi-isomorphisms as the weak equivalences, the fibrations are degreewise projections and the cofibrations are degreewise injections with projective cokernel. The homotopy theory in this setting is the same as regular homological algebra. This also motivates how we will define the model structure on DG-algebras, but we need to be a bit more careful about the construction. 
\end{example}



\subsection{Constructions in model categories}

We said that model structures allows us to introduce the notion of homotopy into the category. We will now see this construction, but first we introduce the definition of the homotopy category. Rather surprisingly, and unintuitively, this has nothing to do with homotopies at all. 

\begin{definition}[The homotopy category]
Let $\mathcal{C}$ be a model category and $W$ its collection of weak equivalences. We define the homotopy category of $\mathcal{C}$ to be the category $Ho\mathcal{C} = \mathcal{C}[W^{-1}]$, i.e. the localization at the weak equivalences. 
\end{definition}

\begin{remark}
The readers familiar with homological algebra will hopefully see some similarities to derived categories of abelian categories. 
\end{remark}

Since a model category is bicomplete, it has both an initial object $I$ and a terminal object $T$. Remember that these are objects where there exists a unique map from and to any other object in the category respectively. 

\begin{definition}[Fibrant object]
Let $X$ be an object in a model category $\mathcal{C}$. We say $X$ is fibrant if the unique map $X\longrightarrow T$ is a fibration. 
\end{definition}

\begin{definition}[Cofibrant object]
Let $X$ be an object in a model category $\mathcal{C}$. We say $X$ is cofibrant if the unique map $I\longrightarrow X$ is a cofibration. 
\end{definition}

If $X$ is both fibrant and cofibrant, we reffer to it as bifibrant\index{Bifibrant}. 



\begin{definition}[Cylinder object]
Let $X$ be an object in a model category $\mathcal{C}$. The cylinder object of $X$, usually denoted $Cyl(X)$, is a factorization of the codiagonal map $\nabla: X\coprod X\longrightarrow X$ into
\begin{equation*}
    X\coprod X\overset{i_0+i_1}\longrightarrow Cyl(X) \overset{p}\longrightarrow X,
\end{equation*} 
where $p$ is a weak equivalence. 

If $X\coprod X\overset{i_1 + i_2}\rightarrow Cyl(X)$ is a cofibration, we call $Cyl(X)$ a good cylinder object, and if in addition $p$ is an acyclic fibration, we call $Cyl(X)$ a very good cylinder object.
\end{definition}

\begin{definition}[Path object]
Given an object $X$ in a model category $\mathcal{C}$ we define the path object of $X$, denoted $Path(X)$ to be factorization of the diagonal map $X \rightarrow X\prod X$ into
\begin{equation*}
     X \overset{i}\rightarrow Path(X) \overset{(p_1,p_2)}\rightarrow X \prod X,
\end{equation*}
where $i$ is a weak equivalence. Similarly to the cylinder object, if $Path(X)\overset{p}\rightarrow X\prod X$ is a fibration, we call $Path(X)$ a good path object, and if in addition $i$ is an acyclic cobration, we call $Path(X)$ a very good path object.
\end{definition}

By the factorization axiom (\textbf{MC4}) every object has at least one very good cylinder object and one very good path object. It can be useful to use these in some cases, but in other cases we can actually be interested in cylinder and path objects that aren't necessarily good, or very good. For example, in the Serre model structure on topological spaces, the standard cylinder object $Cyl(X)=X\times I$ is only a good cylinder object when $X$ is a CW complex. It would sometimes be limiting to not use this standard cylinder when working with homotopies of maps between spaces that are not CW complexes, hence the reason for the bit weaker definition. 

Speaking of homotopies, we now have objects that resemble what we use in the category of topological spaces to define homotopies between maps, so we should also be able to define them in any model category. When we define homotopies in topology, we define them as maps from the cylinder $I\times X$ such that the restriction to the boundary of the cylinder gives us the two maps we are constructing a homotopy between. This is also what motivates how we define it in the general setting for model categories as well, but we need to be a bit more careful. For the below definitions we assume that all objects and all morphisms lie in som model category $\C$

\begin{definition}[Left homotopy]
Given two maps $f,g: X\rightarrow Y$ we define a left homotopy $h:f\sim_L g$ from $f$ to $g$ to be a map $h: Cyl(X)\rightarrow Y$ such that the following diagram commutes
\begin{center}
\begin{tikzcd}[column sep=large, row sep=large]
X \arrow[r, "i_1"] \arrow[rd, "f"'] & Cyl(X) \arrow[d, "h"] & X \arrow[l, "i_2"'] \arrow[ld, "g"] \\
                                    & Y                     &                                    
\end{tikzcd}
\end{center}
\end{definition}

\begin{definition}[Right homotopy]
Given two maps $f,g: X\rightarrow Y$ we define a right homotopy $h:f\sim_R g$ from $f$ to $g$ to be a map $h: X\rightarrow Path(Y)$ such that the following diagram commutes
\begin{center}
\begin{tikzcd}[column sep=large, row sep=large]
  & X \arrow[d, "h"] \arrow[rd, "g"] \arrow[ld, "f"'] &   \\
Y & Path(Y) \arrow[l, "p_1"] \arrow[r, "p_2"']        & Y
\end{tikzcd}
\end{center}
\end{definition}

If the cylinder object used to define the left homotopy is a good cylinder object then we call the homotopy a good left homotopy, and similarly if it is a very good cylinder object we call the homotopy a very good left homotopy. The same goes for the path object used to define the right homotopy, which gives us good right homotopies and very good right homotopies.

The fact that homotopy is an equivalence relation on classes of continuous maps is one of the most important and fundamental properties that homotopy has in the category of topological spaces. Thus it should also be important in the general setting. Before we do that, we note that we can upgrade any left homotopy h to a good left homotopy by factoring the map $X\rightarrow Cyl(X)$ into $X\rightarrow Cyl(X)' \overset{\sigma}\rightarrow Cyl(X)$ by \textbf{MC4}. Then $h\circ \sigma$ will be a good homotopy. If $X$ is fibrant, then we can upgrade it further to a very good left homotopy by using the other factorization on $Cyl(X)\rightarrow X$ to get $Cyl(X)\rightarrow Cyl(X)'\rightarrow X$. This factorization gives us a commutative diagram
\begin{center}
\begin{tikzcd}[column sep=large, row sep=large]
Cyl(X) \arrow[r, "h"] \arrow[d]    & Y \arrow[d] \\
Cyl(X)' \arrow[r] \arrow[ru, "h'"] & T          
\end{tikzcd}
\end{center}
where $T$ is the terminal object. The lift $h'$ comes from \textbf{MC3} and gives us the very good left homotopy that we wanted. 

\begin{lemma}
Let $X$ and $Y$ be objects in a model category $\mathcal{C}$. If $X$ is cofibrant then left homotopy defines an equivalence relation on $Hom(X,Y)$.
\end{lemma}
\begin{proof}
Using $X$ itself as a cylinder object together with the map $f:Cyl(X)=X\rightarrow Y$ as a left homotopy shows that any map $f:X\rightarrow Y$ is left homotopic to itself. It is symmetric since we can compose with the switching map $X\coprod X \rightarrow X\coprod X$ that just switches the components. This gives a homotopy “in the other direction”. Lastly, let $f_1\sim_L f_2$ and $f_2\sim_L f_3$ be good homotopies with cylinder objects being $Cyl(X)$ and $Cyl(X)'$ respectively. Then the pushout of the diagram $Cyl(X)' \leftarrow X \rightarrow Cyl(X)$ defines a new cylinder object and a homotopy $f_1\sim f_3$. Hence the relation is reflexive, symmetric and transitive which is the definition of an equivalence relation.
\end{proof}

Dually we also get the exact same result for right homotopy, but we have to switch from $X$ being cofibrant to $Y$ being fibrant. This is because from a model structure on a category $\mathcal{C}$ we also get a model structure on its opposite category. Here the classes of fibrations and cofibrations are switched, but the weak equivalences stay the same. 

It might feel uneasing that we now have two different concepts of homotopy, which we usually don't have when working in topological spaces. There is a good reason for this, because in both the Serre and the Strøm model structure on topological spaces, all objects are fibrant. Hence, by the next lemma, the existence of right homotopies always implies the existence of left homotopies in the category of topological spaces, which means we don’t ever need to make the distinction between them.

\begin{lemma}
Let $f,g:X\rightarrow Y$ be two maps. If $X$ is cofibrant and $f,g$ are left homotopic then they are right homotopic. Dually, if $Y$ is fibrant and $f,g$ are right homotopic, then they are left homotopic.
\end{lemma}
\begin{proof}
Choose a good cylinder object $X\coprod X \overset{i_1 + i_2}\rightarrow Cyl(X) \overset{j}\rightarrow X$ and let $h:Cyl(X)\rightarrow Y$ be a left homotopy between $f$ and $g$. Choose also a good path object $Y\overset{q}\rightarrow Path(Y) \overset{(p_1, p_2)}\rightarrow Y\prod Y$. We then have a commutative diagram
\begin{center}
\begin{tikzcd}[column sep=large, row sep=large]
X \arrow[r, "q\circ f"] \arrow[d, "i_1"']                       & Path(Y) \arrow[d, "{[p_1, p_2]}"] \\
Cyl(X) \arrow[r, "{[f\circ j, h]}"'] \arrow[ru, "\overline{h}"] & Y\coprod Y                       
\end{tikzcd}    
\end{center}
which has a lift $\overline{h}$ by \textbf{MC3}. 
%Note here that we used the fact that $i_1$ is an acyclic cofibration which we have not proved, but it can be seen by the two out of three property since $X$ is assumed to be cofibrant together with the fact that it is a composition of two cofibrations. 
The composition $h\circ i_2:X\rightarrow Path(X)$ gives a right homotopy between $f$ and $g$ as desired. The dual statement is proved dually.
\end{proof}

We can then finally define the notion of homotopy as follows. 

\begin{definition}[Homtopic maps]
We say two maps $f,g:X\rightarrow Y$ are homotopic, denoted $f\sim g$, if they are both left homotopic and right homotopic.
\end{definition}

This means we can finally define homotopy equivalences. 

\begin{definition}[Homotopy equivalence]
We say a morphism $f\colon X\longrightarrow Y$ is a homotopy equivalence if there exists a morphism $g\colon Y\longrightarrow Y$ such that $f\circ g \sim id_Y$ and $g\circ f \sim id_X$. If there exists a homotopy equivalence between two objects, we call them homotopy equivalent.
\end{definition}

If we now restrict our attention to just the bifibrant objects in a model category, we see that we have a well defined notion of homotopy. It is well defined in the sense that it is an equivalence relation. A question we could ask is when two objects are homotopy equivalent, or more generally, how does this notion of homotopy equivalence relate to weak equivalences? We have a very nice correspondence in this setting, i.e. restricting to the bifibrant objects. 

\begin{theorem}[Generalized Whiteheads theorem]
Two bifibrant objects $X$ and $Y$ in a model category $\C$ are homotopy equivalent if and only if they are weakly equivalent. 
\end{theorem}

We won't cover the proof, but refer to \cite[Theorem 1.2.10.]{hovey}. 

This means that localizing at the weak equivalences also turns homotopy equivalences of bifibrant objects into isomorphisms. If we take the subcategory of bifibrant objects, which we denote $\C_{cf}$, we can form its homotopy category $\C_{cf}\longrightarrow \C_{cf}/\sim$. By the generalized Whitehead theorem this map sends weak equivalences to isomorphisms, and hence it has to factor through its homotopy category $Ho(\C_{cf})$ by general theory about localization. We also have an inclusion $\C_{cf} \longrightarrow \C$ which induces a map on their homotopy categories, i.e. $Ho(\C_{cf})\rightarrow Ho(\C)$. The final piece of the puzzle of having a workable homotopy category will come from the fact that those maps form an equivalence of categories $Ho(\C)\cong \C_{cf}/\sim$, which means that we have a nice definition, and a nice way to work with it.











