\clearpage
\section{Summary and last thoughts}
\label{sec:summary}

So, what have we actually accomplished in this thesis? We have covered a lot of material, but do we leave off with some new insight, or at least a deeper understanding? Did we give a satisfactory answer to the \emph{central question}?

We set out on a journey to uncover---and understand---a special relationship between a DG-algebra and its cohomology. This special relationship, formality, told us exactly when the DG-algebra, and its cohomology, contained the same homotopical information. Along the way we discovered potential information in the DG-algebra, that its cohomology could never get a hold of. This information was stored in the Massey products, and they gave us obstructions to having formality. These Massey products were in general not the only possible obstructions, hence we could not answer the central question by simply checking all possible Massey products. 

We were then facing a cross-road. Do we try something else? Or do we continue pursuing along a similar path? We chose the latter, which led us to develop $\A$-algebras. Using this theory we were able to answer the central question: \textbf{A DG-algebra is formal, if and only if its Merkulov model is again a DG-algebra}. This Merkulov model is explicitly and inductively constructed, so we do in theory have an algorithm to confirm weather or not a given DG-algebra is formal. 

Using the theory we developed---in conjunction with some recent results from the literature---we were able to look back at the failure of the Massey products to perfectly detect formality, and discover a situation where they in fact do just that. This allowed us to consider a certain class of topological spaces---the spaces with Lusternik-Schnirelmann category 1---and prove that they are all formal. This method of coming to that conclusion seems to be a new method, so some new insights have in fact been made. We don't think this thesis developed any more deep insight into the theory, but we hope that we were able to tell a relatively cohesive story---a story about a special relationship between a DG-algebra, and its cohomology. 

Still, there are some insights we would like to have seen, and some results that would have been nice to uncover. The precise relationship between Massey products and the higher products in the Merkulov model are seemingly still a mystery. They are certainly related, and this relationship gets more illuminated and refined over time. It would be interesting to look more deeply into Massey products for $\A$-algebras, as they have a nicer correspondence with the higher operations. Understanding why this relationship works, and how to use it for checking formality would have been interesting to understand.

Also, our result, stating that Massey products are the only obstructions to formality---given that the induced product is trivial---means that the vanishing Massey products neatly join together to form a ``trivial'' $\A$-structure on the cohomology algebra. It would also be interesting to develop a theory for seeing when such an assembly can be made for non-trivial Massey products. Are there certain cases when a given $\A$-structure on the cohomology algebra can be constructed out of representatives for the Massey products in some interesting ways? We unfortunately don't have the time to research these questions, but they would have been interesting to pursue, given a couple more months on the project. 