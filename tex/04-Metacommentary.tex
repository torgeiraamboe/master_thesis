

\section{For experienced readers}

This thesis is not meant to be short and concise---as this is in our opinion not the best way to deeply understand abstract mathematics, which after all is our main goal. For this reason there will be some lengthy explanations and some lengthy calculations in order to properly understand the material at hand. There will also be some intentionally vague sentences in order to build intuition and a more natural feeling for the theory. 

The two first sections in chapter 3---on transferring algebraic structures through homotopy equivalences---are not much needed for the thesis, but they give good motivation for why the definition on an $\A$-algebra is the way it is. The two subsections of chapter 3 on using rooted trees as a visual understanding of the relations in DG-algebras, $\A$-algebras and their morphisms, are also not needed to understand the later results, but they too serve as intuition for the relations, and how to use them. These can easily be omitted by readers who have already seen these concepts. 

If a reader wants to more quickly understand the meat of the thesis---or only go through the most important result---I have added below a list of where to find the central definitions and results. There is also a summary at the end, which quickly summarizes what has been done throughout the thesis. 

\begin{enumerate}
	\item Definition 1.4. (DG-algebra)
	\item Example 1.8. (Cohomology algebra)
	\item Definition 1.15. (Quasi-isomorphic DG-algebras)
	\item Definition 1.16. (Formal DG-algebra)
	\item Definition 2.4. (Massey products)
	\item Definition 2.5. (Vanishing Massey products)
	\item Theorem 2.10. (Formal $\implies$ vanishing Massey products)
	\item Definition 3.3. ($\A$-algebra)
	\item Definition 3.6. ($\A$-quasi-isomorphism)
	\item Theorem 3.10. (Kadeishvili's theorem)
	\item Corollary 3.15. (Trivial $\A$-structure $\implies$ formal)
	\item Theorem 3.23. (Vanishing Massey products + + $\implies$ formal)
	\item Corollary 4.10. ($\ls(X)\leq 1$ $\implies$ $X$ formal)
\end{enumerate}

We will try to explain all the details as they come up, but we do not at all claim that this thesis is self contained. We assume some familiarity with homological algebra and algebraic topology, as well as some general mathematical maturity. 

All images and diagrams are made by the author. 